\documentclass{article}
\usepackage{amsmath, amssymb, amsthm}
\usepackage{geometry}
\usepackage{ctex}
\geometry{a4paper, margin=1in}

\title{第一章\ 线性方程组的求解}
\date{}

\begin{document}

\maketitle

\section{线性方程组的基本概念}

\subsection{一般形式}
设 \(\mathbb{P}\) 是一个数域,\(m, n\) 为正整数,\(n\) 元线性方程组的一般形式为:

\[
\begin{cases}
a_{11}x_{1} + a_{12}x_{2} + \dots + a_{1n}x_{n} = b_{1}, \\
a_{21}x_{1} + a_{22}x_{2} + \dots + a_{2n}x_{n} = b_{2}, \\
\dots \dots \dots \dots \dots \dots \dots \dots \dots \dots \dots \dots \\
a_{m1}x_{1} + a_{m2}x_{2} + \dots + a_{mn}x_{n} = b_{m},
\end{cases}
\]

其中 \(a_{ij}, b_{i} \in \mathbb{P}\) 已知,\(x_{1}, \dots, x_{n}\) 为未知量。

\subsection{相关术语}
\begin{itemize}
    \item \textbf{未知量}:\(x_{1}, x_{2}, \dots, x_{n}\)
    \item \textbf{项}:\(a_{ij}x_{j}\) 为第 \(i\) 个方程的第 \(j\) 项
    \item \textbf{系数}:\(a_{ij}\) 为 \(x_{j}\) 在方程 \(i\) 中的系数
    \item \textbf{常数项}:\(b_{i}\)
    \item \textbf{齐次方程组}:所有 \(b_{i} = 0\)
    \item \textbf{非齐次方程组}:至少一个 \(b_{i} \neq 0\)
\end{itemize}

\subsection{解的概念}
若存在 \(c_{1}, \dots, c_{n} \in \mathbb{P}\) 使得代入后所有方程成立,则称 \((c_{1}, \dots, c_{n})\) 为方程组的一个解。所有解构成\textbf{解集}。方程组有解称为\textbf{相容},否则为\textbf{不相容}。

\subsection{同解方程组}
若两个方程组解集相同,则称它们\textbf{同解}。

\section{同解变形与阶梯形方程组}

\subsection{三类初等变换}
\begin{enumerate}
    \item \textbf{互换}:交换两个方程的位置
    \item \textbf{倍乘}:用非零常数乘某个方程
    \item \textbf{倍加}:将一个方程的常数倍加到另一个方程
\end{enumerate}

\textbf{引理1}:初等变换不改变方程组的解集。

\subsection{阶梯形方程组}
通过初等变换可将方程组化为\textbf{阶梯形},形如:

\[
\begin{cases}
b_{1j_{1}}x_{j_{1}} + \dots + b_{1n}x_{n} = c_{1}, \\
b_{2j_{2}}x_{j_{2}} + \dots + b_{2n}x_{n} = c_{2}, \\
\dots \dots \dots \dots \dots \dots \\
b_{rj_{r}}x_{j_{r}} + \dots + b_{rn}x_{n} = c_{r}, \\
0 = c_{r+1},
\end{cases}
\]

其中 \(b_{ij_{i}} \neq 0\),\(j_{1} < j_{2} < \dots < j_{r}\),且 \(0 = c_{r+1}\) 可能为 \(0 = 0\)(恒等式)或 \(0 = d \ (d \neq 0)\)(矛盾式)。

\textbf{定理1}:任何线性方程组均可通过有限次初等变换化为阶梯形。

\subsection{阶梯头与秩}
阶梯转弯处的项 \(b_{ij_{i}}x_{j_{i}}\) 称为\textbf{阶梯头}。\(r\) 称为方程组的\textbf{秩},代表有效方程的数量。

\section{Gauss 消元法}

\subsection{基本思想}
通过初等变换将方程组化为阶梯形,然后从最后一个方程开始\textbf{回代}求解。

\subsection{解的情况判断}
\begin{enumerate}
    \item 若出现 \(0 = d \ (d \neq 0)\),则方程组无解。
    \item 若 \(r = n\)(有效方程数等于未知量数),则有唯一解。
    \item 若 \(r < n\),则有无穷多解,其中 \(n - r\) 个变量为\textbf{自由未知量}。
\end{enumerate}

\subsection{求解步骤}
\begin{enumerate}
    \item 化为阶梯形
    \item 判断解的存在性与唯一性
    \item 回代求解,用自由未知量表示通解
\end{enumerate}

\section{扩展内容}

\subsection{线性方程组的矩阵表示}
方程组 (1.1.1) 可写为矩阵形式:

\[
A\mathbf{x} = \mathbf{b},
\]

其中 \(A = (a_{ij})_{m \times n}\) 为系数矩阵,\(\mathbf{x} = (x_{1}, \dots, x_{n})^{T}\),\(\mathbf{b} = (b_{1}, \dots, b_{m})^{T}\)。

\subsection{增广矩阵与初等行变换}
对应方程组的初等变换,矩阵可进行\textbf{初等行变换}:
\begin{itemize}
    \item \textbf{互换}:交换两行
    \item \textbf{倍乘}:用非零数乘某一行
    \item \textbf{倍加}:将一行的倍数加到另一行
\end{itemize}
阶梯形方程组对应\textbf{行阶梯形矩阵}。

\subsection{线性方程组的几何意义}
\begin{itemize}
    \item 在 \(\mathbb{R}^{2}\) 中,每个方程表示一条直线,解为直线交点
    \item 在 \(\mathbb{R}^{3}\) 中,方程表示平面,解为平面交点
    \item 无解、唯一解、无穷多解分别对应几何对象的无交点、唯一交点、重合或交于直线/平面
\end{itemize}

\subsection{数值方法简介}
\begin{itemize}
    \item \textbf{高斯消元法}:精确解,适合中小规模稠密矩阵
    \item \textbf{迭代法}(如 Jacobi、Gauss-Seidel):适合大规模稀疏矩阵
    \item \textbf{矩阵分解法}(如 LU 分解):提高计算效率与稳定性
\end{itemize}

\section{总结}
第一章以线性方程组的求解为核心,从基本概念出发,引入初等变换与阶梯形方程组,系统介绍 Gauss 消元法。内容由浅入深,既强调理论严谨性,又注重计算实用性,为后续矩阵、向量空间等概念奠定基础。

\end{document}